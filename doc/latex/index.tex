\begin{DoxyAuthor}{\-Autor}
-\/ \-Maciej \-Lech $<$\href{mailto:lechmac@gmail.com}{\tt lechmac@gmail.\-com}$>$\par
-\/ \-Karol \-Sowa $<$\href{mailto:karol.sowa@gmail.com}{\tt karol.\-sowa@gmail.\-com}$>$ 
\end{DoxyAuthor}
\begin{DoxyDate}{\-Data}
\-Styczeń 2014 
\end{DoxyDate}
\begin{DoxyVersion}{\-Wersja}
1.\-0
\end{DoxyVersion}
\-Program ten powstał jako projekt na laboratorium z {\bfseries \-Obiektowych \-Metod \-Projektowania \-Systemów} na kierunku \-Elektronika i \-Telekomunikacja \-A\-G\-H w \-Krakowie.\par
\par
 \-Program przelicza waluty według kursu pobranego ze strony \href{http://www.nbp.pl}{\tt \-Narodowego \-Banku \-Polskiego}.\par
 \-Pobiera on kwoty do przeliczenia ze standardowego wejścia lub z pliku i zapisuje przeliczone wartości na standardowym wyjściu lub w podanym pliku. \-Dane wyjściowe mogą być różnie formatowane, jako zestawienie tabelaryczne lub lista, z różnymi separatorami do wyboru.\hypertarget{index_etykieta-konstrukcja}{}\section{\-Konstrukcja i wykorzystane wzorce projektowe}\label{index_etykieta-konstrukcja}
\-W dużym uproszczeniu konstrukcję programu można sprowadzić do poniższego schematu\-: 
\begin{DoxyItemize}
\item \-Jak widać, główne trzy etapy zostały zrealizowane za pomocą różnych wzorców projektowych. \-Wspólnym elementem tych etapów jest klasa \hyperlink{class_converter}{\-Converter}, która pełni rolę fabryki we wzorcu {\bfseries factory method} pobierającej dane wejściowe, dyrektora we wzorcu {\bfseries builder} nadzorującego generowanie danych wyjściowych, jak i klasy wybierającej metodę pozyskiwania kursów walutowych opartą na wzorcu {\bfseries strategy}.\par

\item \-Dodatkowo w trakcie tworzenia skorzystano z wzorca {\bfseries singleton} w klasie głównej aplikacji \hyperlink{class_application}{\-Application}, w klasie \hyperlink{class_converter}{\-Converter} oraz w klasie obsługującej powiadomienia \hyperlink{class_log}{\-Log}.\par

\item \-Do pobierania danych przez protokół http wykorzystano wzorzec projektowy {\bfseries adapter}, tworząc wrapper \-C++ biblioteki \-C libcurl.
\end{DoxyItemize}\hypertarget{index_etykieta-korzystanie}{}\section{\-Kompilacja i uruchomienie}\label{index_etykieta-korzystanie}

\begin{DoxyItemize}
\item \-Program był tworzony i testowany z wykorzystaniem\-: \begin{DoxyVerb}
	Ubuntu 12.04 LTS 64 bit
	GCC 4.6.3
	libstd++ 4.6.3
	libcurl 7.22.0
	Code::Blocks IDE 10.05
	doxygen 1.7.6.1
\end{DoxyVerb}

\item \-Zrzut programu uruchomionego z opcją -\/h\-: \begin{DoxyVerb}
Usage:
	currency_calc -ic CURRENCY -oc CURRENCY [OPTIONS]

Options:
	-ic X  --in_curr X	Type of input currency
	-oc X  --out_curr X	Type of output currency

	-e X  --ex_rate X	Exchange rate source (nbp_http*)
	-d X  --date X		Exchange rate date (last*, %RRMMDD%)

	-i X  --in X		Set location of input data (stdin*, %FILE%)
	-o X  --out X		Set location of output data (stdout*, %FILE%)
	-t X  --type X		Type of output data (text_table_space_sep, text_table_tab_sep*, text_space_sep, text_tab_sep, text_newline_sep)

	-h  --help		Show this help message
	-a  --about		Show about
	-v  --verbose		Verbose mode

	*  Default option

\end{DoxyVerb}

\item \-Przykłady uruchomienia programu\-: \begin{DoxyVerb}
	> currency_calc -v -ic pln -oc eur -i ./dane_do_przeliczenia.txt -o ./dane_przeliczone.txt
	> currency_calc -v -ic eur -oc usd -t text_newline_sep
	> currency_calc -v -ic jpy -oc krw -t text_space_sep -d 131024
\end{DoxyVerb}
 
\end{DoxyItemize}